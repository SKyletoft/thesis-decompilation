Figur \ref{fig:tidsplan} visar en preliminär tidsplan för projektet och
aktiviteter för varje läsvecka. Alla externa datum och deadlines för inlämningar
är listade i tabell \ref{tab:deadlines}.


\begin{table}[h]
\caption{Datum och deadlines för viktiga aktiviteter}
\begin{center}
\begin{tabular}{| l | c |}
\hline
\bfseries{Aktivitet}                                 &  \bfseries{Datum} \\
\hline
Projektplan/Planeringsrapport                        &  2023-02-10\\
Muntlig halvtidsredovisning                          &  2023-03-07\\
Egen utvärdering 1                                   &  2023-03-12\\
Slutrapport, förgranskning                           &  2023-04-27\\
Sammanställd slutrapport                             &  2023-05-15\\
Bidragsrapport                                       &  2023-05-16\\
Film inlämning                                       &  2023-05-17\\
Skriftlig individuell opposition                     &  2023-05-22\\
Slutredovisning                                      &  2023-05-25/26\\
Egen utvärdering 2                                   &  2023-05-29\\
Slutlig inlämning av sammanställd slutrapport        &  2023-06-02\\
    \hline
\end{tabular}
\label{tab:deadlines}
\end{center}
\end{table}

\begin{figure}[htp]
\begin{center}

\newganttchartelement{foobar}{
    foobar/.style={
        shape=rectangle,
        inner sep=0pt,
        draw=gray!70!white,
        % draw=black,
        very thick,
        top color=gray!40,
        bottom color=gray!50,
        rounded corners=2pt
    },
    foobar incomplete/.style={
        /pgfgantt/foobar,
        draw=gray!70!white,
        top color=gray!30,
        bottom color=gray!10
    },
    foobar label font=\slshape,
}

\begin{ganttchart}[
        expand chart=\textwidth,
        y unit chart=0.9cm,
        foobar top shift=.3,
        foobar height=.7,
        vgrid={*{8}{gray, dotted}, *1{black, dashed}},
        title/.style={fill=gray, draw=none},
        title label font=\color{white}\bfseries,
        title height=1,
        progress=today,
        progress label text=\relax,
        today=4,
        milestone inline label node/.append style={rotate=30},
        milestone inline label node/.append style={left=2mm},
        milestone top shift=.45,
        milestone right shift=-.2,
        milestone left shift=0.3,
        foobar label font=\footnotesize,
    ]{1}{19}
    \gantttitle{LP 3}{9}        \gantttitle{LP 4}{9}        \gantttitle{}{1}\\
    \gantttitlelist{1,...,9}{1} \gantttitlelist{1,...,9}{1} \gantttitle{}{1} \\

    \ganttfoobar[inline]{Projektplan}{1}{4} \\

    % Eget
    \ganttfoobar[inline]{*\stoe-bygge}{2}{4} \\
    \ganttfoobar[inline]{*\stoe-infra}{3}{5} \\
    \ganttfoobar[inline]{*Demo-förslag}{5}{10} \\
    \ganttfoobar[inline]{*Demo-implementation}{5}{12} \\
    \ganttfoobar[inline]{*GUI-ramverk}{6}{11} \\ \\
    \ganttfoobar[inline]{*Slutprodukt}{8}{15} \\

    \ganttfoobar[inline]{Halvtidsredovisning}{6}{8} \\

    % TODO include deadlines for drafts of report for feedback meetings with fackspråk.
    \ganttfoobar[inline]{Film}{16}{17} \\
    \ganttfoobar[inline]{Skriftlig opposition}{17}{18} \\
    \ganttfoobar[inline]{Slutredovisning}{17}{18} \\

    \ganttfoobar[inline, foobar inline label node/.append style={left=2mm}]{Projektrapport}{5}{19}
    \ganttmilestone[inline]{Förgranskning}{14}{14}
    \ganttmilestone[inline]{Slutrapport}{16}{16}
    \ganttmilestone[inline]{Slutgiltig inlämning}{19}{19} \\

    \ganttlink{elem1}{elem2}
    \ganttlink{elem2}{elem4}
    \ganttlink{elem2}{elem6}
    \ganttlink{elem3}{elem4}
    \ganttlink{elem4}{elem6}
    \ganttlink{elem5}{elem6}

\end{ganttchart}
\end{center}
\caption{Tidsplan}
\label{fig:tidsplan}
\end{figure}

\subsection{Förklaring av aktiviteter i tidsplanen}

Tidsplanen i figur~\ref{fig:tidsplan} innehåller deadlines i italics,
administrativa och rapportrelaterade aktiviteter i normal stil och interna och
produktrelaterade aktiviteter har en stjärna * som prefix. Här följer en
förklaring av de produktrelaterade aktiviteterna:

\begin{labeling}{tidsplansbegrepp}

  \item [\textbf{\stoe-bygge}] Att etablera en miljö där \stoe:s huvudkomponent
    \texttt{libs2e.so} kan byggas reproducibelt och köras tillsammans med QEMU
    på en godtycklig binär, fast utan någon implementerad analys.

  \item [\textbf{\stoe-infra}] Att generera Rust-bindings till \texttt{libs2e.so}.
    Tooling för att kunna köra QEMU+\stoe med ett Rust-plugin på en testbinär.
    Viss abstraktion som gör det ergonomiskt att utveckla Rust-plugin:et.

  \item [\textbf{Demo-förslag}] Att skriftligt föreslå ett \textit{demo}, alltså
    en analys som kan implementeras ovanpå \stoe i form av ett program med
    specifikt syfte att genomföra en specifik sorts analys på små testprogram
    implementerade som enskilda C-filer.

  \item [\textbf{Demo-implementation}] Att implementera demon ovanpå \stoe, utan
    att integrera deras analyser till en sammanhängande enhet.

  \item [\textbf{GUI-ramverk}] Det kringarbete som krävs för att implementera
    analyser i allmänhet i slutprodukten, däribland det kringarbete som krävs
    för att bygga ett grafiskt gränssnitt.

  \item [\textbf{Slutprodukt}] Slutprodukten, med analyser härstammande från
    demon implementerade i en sammanhängande enhet.

\end{labeling}
