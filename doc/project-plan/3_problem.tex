% Det här avsnittet är ofta den viktigaste delen av planeringsrapporten
% (och av den slutgiltiga uppsatsen/rapporten). Den syftar till att
% identifiera frågan/frågorna som ska tas upp i projektet. Det är
% viktigt att gruppen gör en problemanalys även om det i
% projektförslaget redan finns ett problem (en uppgift)
% specificerat. Anledningen till detta är att det riktiga primära
% problemet ofta skiljer sig från det i början av
% uppdragsgivaren/förslagsställaren/kunden föreslagna. Problemanalysen
% syftar också till att bryta ner problemet/uppgiften i mindre och mer
% detaljerade delproblem/deluppgifter, vilket också leder till
% formulering av delsyften. Genom att göra detta får studenterna mycket
% bättre förståelse för de olika aspekterna av problemet/uppgiften. Utan
% den här informationen är det omöjligt att identifiera vilken
% information som behövs, vilka informationskällor som behövs och
% lämpliga tillvägagångssätt.

% En bra problemanalys som identifierar delproblem/deluppgifter och
% delsyften vilar i många fall på användning av teorier och modeller
% från litteraturen. En litteraturgenomgång bör därför genomföras tidigt
% i processen.

För att uppnå projektets syfte måste vi dela upp problemet i mindre
delar. Första delen är elf-tolkning och dekompilering. Sedan symbolisk
exekvering av det tolkade programmet.

När vi väl har ett tolkat och körbart program kan vi sedan spåra
exekveringen och skapa en kontrollflödesgraf som kan användas till att
återskapa större strukturer som for- eller while-loopar, if-satser
eller virtuella funktionsanrop.

Grafen kan också visualiseras för att hjälpa användaren förstå
programmet. Alternativt kan hela programmet återskapas som C-kod.

För analys av programmet är det också hjälpsamt att kunna stega genom
exekveringen steg för steg och ändra på värden för att ta de vägar man
vill analysera. Detta borde också kunna göras baklänges, dvs att man
väljer en slutdestination och låter programmet själv lista ut vilka
värden som behövdes läsas för att exekveringen skulle ta sig till den
punkten.

% Symbolic execution engine
% S2E
% state-merging
