\subsection{Val av strategi} 
För att uppnå och redogöra för möjligheterna i en
potentiell applikation reflekterades det över två möjliga alternativ att
fullfölja. Dels diskuterades alternativet att bygga en symbolic-execution engine
från grunden och fokusera på dess tekniska detaljer och dels att använda en
existerande produkt där fokuset istället ligger på att bygga plugins vars syfte
är att utöka den existerande motorn med fler funktioner. 

Med syfte att göra framsteg valdes det att kolla vidare hur S2E kan hjälpa i
frågan om att utöka funktionalitet i en existerande motor. S2E är utvecklat
i programspråket C/C++ och med anledning av valet att utveckla i ett
annat programmeringsspråk, det vill säga Rust, krävs implementation av
C/C++-bindings. I ett första steg valdes det därför att se över hur detta går till
på lämpligt sätt och hur detta kan automatiseras med hjälp av andra verktyg
där bland annat autocxx ansågs som ett lovande alternativ. 

\subsection{Tillvägagångssätt} 
I avsikt att uppnå syftet med projektet, mer
specifikt att utveckla en applikation, bestämdes det att utveckla demon av enkla
program skrivna i C med hjälp av bindings till S2E som implementerats för att
undersöka potentiella användningsområden och utveckla dessa vidare.
Denna metod är önskad eftersom det tillåter att sätta uppnåbara delmål som styr
funktionaliteter som ska implementeras och udvidgning av verktyget som ska ske
härnäst. Dessutom kan flera demon utvecklas samtidigt vilket tillåter
parallellism inom projektarbetet och framsteg på flera fronter.

% utveckla detta (konkretisera etc.)
För att vidare kunna bestämma huruvida en demo faller inom ett lämpligt
användningsområde kommer en avgränsning ske efter vad som är rimligt; hur väl
kan det komma att tillämpas i en applikation; vad är relevant att undersöka med
applikationen samt om det är genomförbart inom satt tidsram. 

I ett senare och/eller parallellt steg ska ett intuitivt grafiskt användargränssnitt
utvecklas som tillåter användaren att traversera genom applikationen och göra
egna beslut gällande förgreningar av programmet etc. där användaren själv
bestämmer interaktivt nästa beslut som görs. Även detta arbetet styrs av demon
och funktionaliteter som önskas.

% Hur är detta kopplat till vårt syfte? Hur uppnår vi syftet med rapporten genom
% vald metod?

% (varför är typ besvarat i val av strategi-avsnittet)

% Hur besvarar vi huruvida valet av strategi är rimligt?

%  

