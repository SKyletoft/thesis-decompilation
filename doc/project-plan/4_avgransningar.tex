Att skapa en emulator är tidskrävande, dessutom är den resulterande produktens
korrekthet inte tillförlitlig utan nogrann testning. För att fastställa
korrekthet ska applikationen därför utnyttja existerande verktyg för att
exekvera programmet med stöd för symbolisk exekvering. Detta möjliggör också
bredare plattformssupport jämfört med en hemmasnickrad emulator.

\subsection{\stoe}

\stoe\cite{s2e} är en plattform för symbolisk exekvering som bygger på QEMU:s
virtuella maskin och använder KLEE\cite{klee}, en motor för symbolisk
exekvering, som interpreter för att möjliggöra symbolisk exekvering. \stoe\ är i
sin tur utbyggbart med möjlighet för användaren att skriva ett eget plugin för
att utföra analyser och används inom säkerhetsforskning för att till exempel
analysera skadlig kod. \stoe\ användes som del av Galactica-systemet som spelade
i DARPA Cyber Grand Challenge.\cite{s2ecgc} \stoe\ är open-source,
väldokumenterat och underhålls aktivt.

\subsection{SymQEMU}

Ett alternativt verktyg för att symbolisk exekvering är symQEMU\cite{symqemu},
som också kombinerar QEMUs virtuella maskin med KLEE:s motor för symbolisk
exekvering. Till skillnad från \stoe\ kompilerar SymQEMU KLEE in i den
analyserade binären och har jämförelsevis hög prestanda. Däremot har SymQEMU
bristfällig dokumentation och är ej aktivt uppdaterat.

\subsection{Beslut}

Då SymQEMU ej uppdateras aktivt och har bristfällig dokumentation kommer \stoe\
användas. Projektet avgränsas i och med att existerande verktyg (\stoe) kommer
användas istället för att bygga en motor för symbolisk exekvering från grunden.

\subsection{Konsekvenser}

Att använda \stoe\ innebär att arbetet avgränsas till att skapa ett plugin som
bygger ut motorn. Varken emulator eller motor ska byggas och de uppgifter som
ingår i att skapa en exekveringsmotor exkluderas.

Avgränsningen medför dessutom att fokus flyttas ifrån motorns tekniska detaljer
till att utveckla en användbar slutprodukt som bygger ut \stoe:s redan
existerande funktionalitet med ett grafiskt användargränsnitt och möjlighet att
stega igenom, analysera och interaktivt besluta om värden under exekvering.

Beslutet innebär att applikationens utformning blir bunden till \stoe:s tekniska
begränsningar.
